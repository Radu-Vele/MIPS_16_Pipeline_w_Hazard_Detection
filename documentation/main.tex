\documentclass[a4paper,12pt]{report}

% Language setting
% Replace `english' with e.g. `spanish' to change the document language
\usepackage[english]{babel}

% Set page size and margins
% Replace `letterpaper' with`a4paper' for UK/EU standard size
\usepackage[a4paper,top=2cm,bottom=2cm,left=2cm,right=2cm,marginparwidth=1.75cm]{geometry}

% Useful packages
\usepackage{algorithmic}
\usepackage{amsmath}
\usepackage{graphicx}
\usepackage{listings} 
\usepackage{color}
\usepackage{xcolor}
\usepackage{hyperref}
\usepackage{verbatim}

% Diagrams
\usepackage{tikz}
\usetikzlibrary{positioning}

\usepackage{titlesec}
\titleformat{\chapter}{}{}{}{\bf\LARGE}

% this is needed for forms and links within the text
\usepackage{hyperref}


\title{\textbf{Hazard Detection and Avoidance Unit}}
\author{
    \IEEEauthorblockN{student: Radu-Augustin Vele} \\\\
    \hline\\
    \IEEEauthorblockA{{Structure of Computer Systems Project} \\ \\
    \hline \\
    {$3^{rd}$ year student} \\
    {Technical University of Cluj-Napoca}}} 
    
\begin{document}

\maketitle

\newpage

\tableofcontents

\pagenumbering{gobble}

\newpage

\pagenumbering{arabic}

\chapter{Introduction}
\section{Context} 

The aim of this project is to provide a Hazard Detection and Avoidance unit for the MIPS (\textbf{M}icroprocessor without \textbf{I}nterlocked \textbf{P}ipeline \textbf{S}tages) 16 Pipeline architecture. Taking advantage of pipelining in the design of a CPU brings multiple benefits, but comes with the cost of possibly occurring hazard conditions. This decreases the reliability of the computer system depending on that CPU.


A hazard detection and avoidance unit improves the design by identifying and solving some of the more frequent hazard conditions. The result is a reliable synthesizable design that can be programmed on an FPGA and used for building personalized computer systems for specific use cases.
 
\section{Objectives}

The unit will be designed in VHDL and included in a Xilinx Vivado Project containing an already designed MIPS 16 Pipeline. The result of the design is a bitstream that can be programmed on a Basys 3 board (designed by Digilent). 

Besides the general execution of instructions, the final MIPS will detect and resolve three types of hazard conditions:

\begin{itemize}
    \item Read-After-Write (data hazard): dependencies of instructions on data that is not yet available.
    \item Load data hazard (data hazard): a special case where stalls (bubbles) need to be inserted.
    \item Control hazards (branch hazards): situations where the flow of the program is not the intended one as several instructions that would not be executed after a branch are already in the pipeline
\end{itemize}

\chapter{Bibliographic Research}
\section{What is a hazard?}
The MIPS architecture has been designed to include 5 pipeline stages: Instruction Fetch, Instruction Decode, Execution, Memory and Write Back. The hazards occur when the instruction that is next in line can not be executed properly and can be divided into three categories~\cite{patterson2014computer}: \textbf{structural} hazards, \textbf{data} hazards, and \textbf{control} hazards.

In this project the focus is on the last two types of hazards as the structural ones are a concern only when more instructions need to access the same resource at the same time. It is not the case for the MIPS 16 as the only such resource would be the memory in the case the Data and Instruction Memories would not be separated. 

Moreover, only the hardware solutions to hazards are explored as there is also the option to use software approaches for solving them (usually involving the compiler, e.g. reordering instructions)

\section{Solving data hazards}
Data hazards occur when the instruction that is about to be executed needs some data that is not ready yet. This could be easily, but inefficiently solved by inserting stalls (instructions with no effect) until all data is ready. A better solution is designing a \textbf{forwarding unit}~\cite{patterson2014computer} that fast-forwards the data to where is needed immediately after is ready. This implies the need for additional control signals that allow data to be forwarded only when necessary.

However, in the case of our chosen architecture there is a special case that involves an instruction that needs data loaded from the memory in the previous instruction. In this case one stall \textbf{needs} to be inserted. Thus, a \textbf{hazard detection unit}~\cite{patterson2014computer} is needed. It has to be able to detect the special condition and insert a nop instruction.

\section{Solutions for control hazards}
Control hazards generally involve the branch instruction. As the outcome of the branch is only known in a later stage in the pipeline (Execution or Memory stage). Therefore, at each clock cycle between the start of the branch and the result of its test a new instruction is introduced in the pipeline. If the branch condition is true those instructions must not have an effect to have a correct program. Therefore they need to be \textbf{flushed}~\cite{patterson2014computer}. The solution chosen is \textbf{assume branch not taken}~\cite{patterson2014computer} meaning that we guess that the branch test is not true and the next instructions will be executed, if our guess is wrong we discard those instructions. To reduce the number of instructions to be flushed we modify the architecture such that the branch decision is computed in Instruction Decode state. Further improvements involving dynamic branch prediction ~\cite{articleBP} and branch delay slot will not be implemented in this design as they require additional software support.

% - data forwarding meaning
% - hazard detection and stalls meaning
% - control hazards - assume not taken and improvement of the pitfall (delay slot, ...)

% mention what they mean, how the solutions for them have been designed, not into great detail, example diagrams and stuff

\chapter{Analysis}

\chapter{Design}

\chapter{Implementation}

\chapter{Testing \& Validation}

\chapter{Conclusions}

\addcontentsline{toc}{chapter}{Bibliography}
\bibliographystyle{IEEEtran}
\bibliography{sample}

\end{document}






% Examples on how to use the IEEE template

% \section{Introduction}

% \begin{equation}
    
% \end{equation}

% Your introduction goes here! Simply start writing your document and use the Recompile button to view the updated PDF preview. Examples of commonly used commands and features are listed below, to help you get started.

% Once you're familiar with the editor, you can find various project setting in the Overleaf menu, accessed via the button in the very top left of the editor. To view tutorials, user guides, and further documentation, please visit our \href{https://www.overleaf.com/learn}{help library}, or head to our plans page to \href{https://www.overleaf.com/user/subscription/plans}{choose your plan}.

% \section{Some examples to get started}

% \subsection{How to create Sections and Subsections}

% Simply use the section and subsection commands, as in this example document! With Overleaf, all the formatting and numbering is handled automatically according to the template you've chosen. If you're using Rich Text mode, you can also create new section and subsections via the buttons in the editor toolbar.

% \subsection{How to include Figures}

% First you have to upload the image file from your computer using the upload link in the file-tree menu. Then use the includegraphics command to include it in your document. Use the figure environment and the caption command to add a number and a caption to your figure. See the code for Figure \ref{fig:frog} in this section for an example.

% Note that your figure will automatically be placed in the most appropriate place for it, given the surrounding text and taking into account other figures or tables that may be close by. You can find out more about adding images to your documents in this help article on \href{https://www.overleaf.com/learn/how-to/Including_images_on_Overleaf}{including images on Overleaf}.

% % \begin{figure}
% % \centering
% % \includegraphics[width=0.3\textwidth]{frog.jpg}
% % \caption{\label{fig:frog}This frog was uploaded via the file-tree menu.}
% % \end{figure}

% \subsection{How to add Tables}

% Use the table and tabular environments for basic tables --- see Table~\ref{tab:widgets}, for example. For more information, please see this help article on \href{https://www.overleaf.com/learn/latex/tables}{tables}. 

% \begin{table}
% \centering
% \begin{tabular}{l|r}
% Item & Quantity \\\hline
% Widgets & 42 \\
% Gadgets & 13
% \end{tabular}
% \caption{\label{tab:widgets}An example table.}
% \end{table}

% \subsection{How to add Comments and Track Changes}

% Comments can be added to your project by highlighting some text and clicking ``Add comment'' in the top right of the editor pane. To view existing comments, click on the Review menu in the toolbar above. To reply to a comment, click on the Reply button in the lower right corner of the comment. You can close the Review pane by clicking its name on the toolbar when you're done reviewing for the time being.

% Track changes are available on all our \href{https://www.overleaf.com/user/subscription/plans}{premium plans}, and can be toggled on or off using the option at the top of the Review pane. Track changes allow you to keep track of every change made to the document, along with the person making the change. 

% \subsection{How to add Lists}

% You can make lists with automatic numbering \dots

% \begin{enumerate}
% \item Like this,
% \item and like this.
% \end{enumerate}
% \dots or bullet points \dots
% \begin{itemize}
% \item Like this,
% \item and like this.
% \end{itemize}

% \subsection{How to write Mathematics}

% \LaTeX{} is great at typesetting mathematics. Let $X_1, X_2, \ldots, X_n$ be a sequence of independent and identically distributed random variables with $\text{E}[X_i] = \mu$ and $\text{Var}[X_i] = \sigma^2 < \infty$, and let
% \[S_n = \frac{X_1 + X_2 + \cdots + X_n}{n}
%       = \frac{1}{n}\sum_{i}^{n} X_i\]
% denote their mean. Then as $n$ approaches infinity, the random variables $\sqrt{n}(S_n - \mu)$ converge in distribution to a normal $\mathcal{N}(0, \sigma^2)$.


% \subsection{How to change the margins and paper size}

% Usually the template you're using will have the page margins and paper size set correctly for that use-case. For example, if you're using a journal article template provided by the journal publisher, that template will be formatted according to their requirements. In these cases, it's best not to alter the margins directly.

% If however you're using a more general template, such as this one, and would like to alter the margins, a common way to do so is via the geometry package. You can find the geometry package loaded in the preamble at the top of this example file, and if you'd like to learn more about how to adjust the settings, please visit this help article on \href{https://www.overleaf.com/learn/latex/page_size_and_margins}{page size and margins}.

% \subsection{How to change the document language and spell check settings}

% Overleaf supports many different languages, including multiple different languages within one document. 

% To configure the document language, simply edit the option provided to the babel package in the preamble at the top of this example project. To learn more about the different options, please visit this help article on \href{https://www.overleaf.com/learn/latex/International_language_support}{international language support}.

% To change the spell check language, simply open the Overleaf menu at the top left of the editor window, scroll down to the spell check setting, and adjust accordingly.

% \subsection{How to add Citations and a References List}

% You can simply upload a \verb|.bib| file containing your BibTeX entries, created with a tool such as JabRef. You can then cite entries from it, like this: \cite{greenwade93}. Just remember to specify a bibliography style, as well as the filename of the \verb|.bib|. You can find a \href{https://www.overleaf.com/help/97-how-to-include-a-bibliography-using-bibtex}{video tutorial here} to learn more about BibTeX.

% If you have an \href{https://www.overleaf.com/user/subscription/plans}{upgraded account}, you can also import your Mendeley or Zotero library directly as a \verb|.bib| file, via the upload menu in the file-tree.

% \subsection{Good luck!}

% We hope you find Overleaf useful, and do take a look at our \href{https://www.overleaf.com/learn}{help library} for more tutorials and user guides! Please also let us know if you have any feedback using the Contact Us link at the bottom of the Overleaf menu --- or use the contact form at \url{https://www.overleaf.com/contact}.

% \bibliographystyle{alpha}
% \bibliography{sample}

% \end{document}